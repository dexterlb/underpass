\documentclass[10pt]{beamer}
\usepackage[english,bulgarian]{babel}
\input{./macros.tex}
\usetheme{Copenhagen}
\usecolortheme{dove}
\title{Катег\'орийни граматики за геопространствени заявки}
\date{\today}
\author{Бла Бла}
\institute{Магистър университет ФМИ бла}
\begin{document}
  \maketitle

  \section{Геопространствени заявки}
  \begin{frame}[fragile]
    Вход:
    \begin{lstwrap}\begin{lstlisting}
      pharmacies near parking spaces in Berlin
    \end{lstlisting}\end{lstwrap}
    Изход:

    \includegraphics[width=0.32\textwidth]{map/world.png}
    \hfill
    \includegraphics[width=0.32\textwidth]{map/berlin.png}
    \hfill
    \includegraphics[width=0.32\textwidth]{map/pharmacy.png}
  \end{frame}

  \section{Overpass}
  \begin{frame}[fragile]{Overpass код}
    \begin{lstwrap}\begin{lstlisting}
    ( node["amenity" = "parking_space"]; ) -> .x1;
    ( area["name" = "Berlin"]; area["int_name" = "Berlin"]; area["name:en" = "Berlin"]; ) -> .x2;
    ( node(area.x2)(around.x1:100.0)["amenity" = "pharmacy"]; ) -> .x3;
    .x3 out;
    \end{lstlisting}\end{lstwrap}
  \end{frame}

  \begin{frame}{Типове обекти в Overpass}
    \begin{itemize}
      \item node
      \item way
      \item relation
      \item area
    \end{itemize}
  \end{frame}

  \begin{frame}{Set}

  \end{frame}

  \begin{frame}[fragile]{Резултат от заявка}
    \begin{lstwrap}\begin{lstlisting}
        out <input set>;
    \end{lstlisting}\end{lstwrap}
  \end{frame}

  \begin{frame}[fragile]{Извличане на обекти}
    \begin{itemize}
      \item Синтаксис
        \begin{lstwrap}\begin{lstlisting}
            <object keyword><filter><filter>...<filter>
        \end{lstlisting}\end{lstwrap}
      \item Пример
        \begin{lstwrap}\begin{lstlisting}
            node[name="Foo"];
            out;
        \end{lstlisting}\end{lstwrap}
        \begin{lstwrap}\begin{lstlisting}
            node[name="Foo"] -> ._;
            out ._;
        \end{lstlisting}\end{lstwrap}
    \end{itemize}
  \end{frame}

  \begin{frame}[fragile]{Извличане на обекти}
    \begin{itemize}
      \item Синтаксис
        \begin{lstwrap}\begin{lstlisting}
            <keyword><filter><filter>...<filter>
        \end{lstlisting}\end{lstwrap}
      \item Ключови думи
        \begin{center}
            \begin{tabular}{r|l}
                \code{node} & retrieves nodes \\
                \code{way} & retrieves ways \\
                \code{relation} & retrieves relations \\
                \code{area} & retrieves areas \\
                \code{nwr} & retrieves nodes, ways, and relations \\
            \end{tabular}
        \end{center}
    \end{itemize}
  \end{frame}

  \begin{frame}[fragile]{Други филтри за ``тагове''}
    \begin{lstwrap}\begin{lstlisting}
        node[amenity="cafe"];
    \end{lstlisting}\end{lstwrap}

    \begin{lstwrap}\begin{lstlisting}
        node[amenity="cafe"][name="Moondeers"];
    \end{lstlisting}\end{lstwrap}

    \begin{lstwrap}\begin{lstlisting}
        node[amenity="cafe"][name~"^moondeers$",i];
    \end{lstlisting}\end{lstwrap}

    \begin{lstwrap}\begin{lstlisting}
        area["administrative"];
    \end{lstlisting}\end{lstwrap}
  \end{frame}

  \begin{frame}[fragile]{Сечение}
      \begin{lstwrap}\begin{lstlisting}
          node[amenity="cafe"] -> .cafes;
          node.cafes;
      \end{lstlisting}\end{lstwrap}
      \begin{lstwrap}\begin{lstlisting}
          node[amenity="cafe"] -> .cafes;
          node.cafes[name="Moondeers"];
      \end{lstlisting}\end{lstwrap}
  \end{frame}

  \begin{frame}[fragile]{Обекти в area}
    \begin{lstwrap}\begin{lstlisting}
      area[name="Frankfurt"] -> fr;
      node(area.fr);
      out;
    \end{lstlisting}\end{lstwrap}
  \end{frame}

  \begin{frame}[fragile]{Обекти на разстояние от други обекти}
    \begin{lstwrap}\begin{lstlisting}
        node[amenity="parking_space"] -> pspaces;
        node[amenity="cafe"](around.pspaces:120) -> x;
        out x;
    \end{lstlisting}\end{lstwrap}
    \begin{lstwrap}\begin{lstlisting}
        area[name="Bonn"];
        node(area)[highway=bus_stop];
        node(around:100)[amenity=cinema];
        out;
    \end{lstlisting}\end{lstwrap}
  \end{frame}

  \begin{frame}[fragile]{Обединение}
    \begin{lstwrap}\begin{lstlisting}
        node[amenity="cafe"] -> .cafes;
        area[name="Bonn"] -> .bonn;
        area[name="Frankfurt"] -> .frankfurt;
        node.cafes(area.bonn) -> .cb;
        node.cafes(area.frankfurt) -> .cf;
        (.cf; .cb;) -> .cfb;
        out .cfb;
    \end{lstlisting}\end{lstwrap}

    \begin{lstwrap}\begin{lstlisting}
        area[name="Bonn"] -> .bonn;
        area[name="Frankfurt"] -> .frankfurt;
        ( node[amenity="cafe"](area.bonn); node[amenity="cafe"](area.frankfurt); );
        out;
    \end{lstlisting}\end{lstwrap}
  \end{frame}

  \section{Категорийни граматики}
  \begin{frame}{Категорийни граматики: пример}
    \gramshort{
        \gramrow{pharmacies}{ GSet }{}
        \gramrow{Berlin}{ GSet }{}
    }
  \end{frame}
  \begin{frame}{Категорийни граматики: пример}
    \gramshort{
        \gramrow{pharmacies}{ GSet }{}
        \gramrow{Berlin}{ GSet }{}
        \gramrow{parking}{ GSet \rc Spaces }{}
        \gramrow{spaces}{Spaces}{}
    }
  \end{frame}
  \begin{frame}{Категорийни граматики: пример}
    \gramshort{
        \gramrow{pharmacies}{ GSet }{}
        \gramrow{Berlin}{ GSet }{}
        \gramrow{parking}{ GSet \rc Spaces }{}
        \gramrow{spaces}{Spaces}{}
        \gramrow{in}{ (GSet \lc GSet) \rc GSet }{}
        \gramrow{near}{ (GSet \lc GSet) \rc GSet }{}
    }
  \end{frame}
  \begin{frame}{Категорийни граматики: пример}
    \autoscaledtree{.{$GSet$}
        [ .{$GSet$}
            [ .{$GSet$} [ .{$pharmacies$} ] ]
            \edge[very thick];
            [ .{$GSet \lc GSet$}
                \edge[very thick];
                [ .{$(GSet \lc GSet) \rc GSet$} [ .{$near$} ] ]
                [ .{$GSet$}
                    \edge[very thick];
                    [ .{$GSet \rc Spaces$} [ .{$parking$} ] ]
                    [ .{$Spaces$} [ .{$spaces$} ] ]
                ]
            ]
        ]
        \edge[very thick];
        [ .{$GSet \lc GSet$}
            \edge[very thick];
            [ .{$(GSet \lc GSet) \rc GSet$} [ .{$in$} ] ]
            [ .{$GSet$} [ .{$Berlin$} ] ]
        ]
    }
  \end{frame}

  \section{Minipass: междинен език}
  \begin{frame}{Графова абстракция}
    \begin{itemize}
      \item Върхове: Overpass обекти
        \begin{itemize}
          \item по тип
          \item по стойност на таг (например име)
          \item ...
        \end{itemize}
      \item Ребра: връзки между обектите \pika
        \begin{itemize}
          \item близост
          \item физическо съдържане
          \item ...
        \end{itemize}
    \end{itemize}
  \end{frame}

  \begin{frame}[fragile]{Графова абстракция: пример}
    Заявка:
    \begin{lstwrap}\begin{lstlisting}
      bus stops near schools in Russia
    \end{lstlisting}\end{lstwrap}

    \begin{enumerate}
      \item Нека $B$ е множеството от върховете, имащи етикет
        ``е автобусна спирка''
      \item Нека $S$ е множеството от върховете, имащи етикет
        ``е училище''
      \item Нека $R$ е множеството от върховете, имащи етикет
        ``английското му име е \emph{Russia}''
      \item Нека $N$ е множеството от върховете, които можем да
        достигнем от върхове в $S$, ходейки по ребро, имащо етикет
        ``е близо до''
      \item Нека $P$ е множеството от върховете, които можем да
        достигнем от върхове в $R$, ходейки по ребро, имащо етикет
        ``е във вътрешността на''
      \item Резултатът е $B \cap N \cap P$.
    \end{enumerate}
  \end{frame}

  \begin{frame}{Синтаксис: Основни типове}
    \begin{itemize}
      \item $Num$ - цяло число \pika
      \item $String$ - низ \pika
      \item $List$ - полиморфен свързан списък, който може
        да съдържа $Num$, $String$ и $List$
      \item $GSet$ - множество от географски обекти: съответства на \code{Set}
        в Overpass
    \end{itemize}
  \end{frame}

  \begin{frame}{Синтаксис: $\lambda$}
    \begin{itemize}
      \item Прилагане на термове (ляво-асоциативно)
        \begin{center}
          \code{<терм1> <терм2>}
        \end{center}
      \item Абстракция
        \begin{center}
          \code{lambda <име на променлива> : <тип> => <терм>}
        \end{center}
      \item Променливи и константи
      \item Сложни типове
        \begin{center}
          \code{<тип1> -> <тип2>}
        \end{center}
    \end{itemize}
  \end{frame}

  \section{Подтипове}
  \begin{frame}{Подтипове}няколко слайда тук\end{frame}

  \section{Категорийни граматики върху Minipass}
  \begin{frame}{Категорийни граматики върху Minipass}няколко слайда тук\end{frame}

  \section{Превод от Minipass към Overpass}
  \begin{frame}{Превод}няколко слайда тук\end{frame}

  \section{Резултат}
  \begin{frame}{Резултат}няколко слайда тук\end{frame}

\end{document}
