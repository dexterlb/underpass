\documentclass[main.tex]{subfiles}
\begin{document}

\subsection{\hsout{Hacks} Extensions of the CCG formalism}

The CCG formalism described in \ref{sec:ccg} is too weak to represent most
phenomena in natural languages \cite{steedman}. Thus, several extensions have
been developed to deal with them.

\subsubsection{Categorial variables}
This is a very powerful extension which adds simple unification based on variables
to categorial derivation. To define it formally, it is convenient to assume
the existence of an infinitely countable set $\mathbb{V}$ whose elements
we call \emph{variables}\footnote{That set is magical in the sense that it is
    disjoint with every set we feel convenient that it be disjoint with.}
and introduce variable substitution over categories.

\begin{defn}
    We extend the definition of categorial closure to include variables:
    \begin{enumerate}
        \item \label{itm:atomic} $A \in \tau \Rightarrow A \in C(\tau)$
        \item \label{itm:right}  $X, Y \in C(\tau) \Rightarrow \lp X \rc Y \rp \in C(\tau)$
        \item \label{itm:left}   $X, Y \in C(\tau) \Rightarrow \lp X \lc Y \rp \in C(\tau)$
        \item \label{itm:var}    $\alpha \in \mathbb{V} \Rightarrow \alpha \in C(\tau)$
    \end{enumerate}
\end{defn}

\begin{defn}
    The function $fv: C(\tau) \rightarrow 2^{\mathbb{V}}$ gives us the set of
    all variables used in a category:
    \[
        fv(Z) =
        \begin{cases*}
            \varnothing, & $Z \in \tau$ \\
            fv(X) \cup fv(Y), & $Z = \lp X \rc Y \rp$ \\
            fv(X) \cup fv(Y), & $Z = \lp X \lc Y \rp$ \\
            \{ \alpha \}, & $Z = \alpha \in \mathbb{V}$ \\
        \end{cases*}
    \]
\end{defn}

\subsubsection{Type-raising}
This extension adds two unary rules for constructing derivations: \cite[sec.~5.3.1]{nts}
\begin{center}
    \tree{.{$\lb \alpha \rc \lp \alpha \lc X \rp \rb$} \edge[very thick]; {$\lb X \rb$} }
        ( Forward type-raising )
    \tree{.{$\lb \alpha \lc \lp \alpha \rc X \rp \rb$} \edge[very thick]; {$\lb X \rb$} }
        ( Backward type-raising )
\end{center}

\end{document}
