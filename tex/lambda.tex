\documentclass[main.tex]{subfiles}
\begin{document}

\subsection{Simply typed $\lambda$-calculus}

For representing compositional terms, a version of simply typed lambda calculus
will be used.

\subsubsection{Types}
\begin{defn}
    Let $T$ be a partially ordered
    set with a comparison operator $\less$.
    Its elements shall be called \emph{basic types}.
    Moreover, let there be a operator $\meet : T \times T \nrightarrow T$,
    such that:
    \begin{itemize}
        \item $\sigma \meet \tau = \nu \iff \tau \meet \sigma = \nu$
        \item $\sigma \meet \tau = \nu \implies \nu \less \sigma$
        \item $\sigma \meet \tau = \nu \implies \nu \less \tau$
        \item $\sigma \in T       \implies \sigma \meet \sigma = \sigma$
    \end{itemize}
\end{defn}

\begin{defn}
    For a set of basic types $T$, its \emph{applicative closure} is defined inductively:
    \begin{itemize}
        \item $\wildcard \in App(T)$
        \item $\sigma \in T \implies \sigma \in App(T)$
        \item $\sigma \in App(T), \tau \in App(T) \implies \sigma \tot \tau \in App(T)$
    \end{itemize}

    Moreover, the $\less$ operator could be defined as follows:
    \begin{itemize}
        \item For $\sigma \in T, \tau \in T$, $\sigma \less \tau$ is already defined.
        \item For $\sigma', \sigma'', \tau', \tau'' \in App(T)$,
                  $\sigma' \tot \tau' \less \sigma'' \tot \tau''
                  \Leftrightarrow (\sigma' \less \tau') \land (\sigma'' \less \tau'')$
        \item For $\sigma \in App(T)$, $\sigma \less \wildcard$
    \end{itemize}


    And the $\meet$ operator could be defined as follows:
    \begin{itemize}
        \item For $\sigma \in T, \tau \in T$, $\sigma \meet \tau$ is already defined.
        \item For $\sigma', \sigma'', \tau', \tau'' \in App(T)$,
                  $\sigma' \tot \tau' \meet \sigma'' \tot \tau''
                  = (\sigma' \meet \tau') \tot (\sigma'' \meet \tau'')$
        \item For $\sigma \in App(T)$, $\wildcard \meet \sigma = \sigma$
                                   and $\sigma \meet \wildcard = \sigma$
    \end{itemize}
\end{defn}

\subsubsection{Terms}
\begin{defn}
    Let $\mathbb{V}$ be an infinite countable set. We shall call its elements
    \emph{variables}.
\end{defn}

\begin{defn}
    Let $C$ be a countable set whose elements we call \emph{constants}.

    The set $\Lambda_T^C$ is defined inductively:

\end{defn}

\end{document}
