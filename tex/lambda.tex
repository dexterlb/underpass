\documentclass[main.tex]{subfiles}
\begin{document}

\subsection{Simply typed $\lambda$-calculus}
\label{sec:lambda}

For representing compositional terms, a version of simply typed lambda calculus
will be used.

\subsubsection{Types}
\begin{defn}
    A \emph{partial meet-semilattice} is a pair $\langle X, \less \rangle$ where:
    \begin{itemize}
        \item $\less$ is a partial-order relation
        \item for any $x, y \in X$, there is at most one $z$ such that
        \[  z \less x, z \less y,
            \forall t \in X (t \less x, t \less y \implies t \less z) \]
        We can define the partial function $\meet : X \times X \leadsto X$ to
        be $x \meet y = z$ for any $x, y, z$ satisfying this condition.
    \end{itemize}
\end{defn}

\begin{defn}
    Let $\mathbb{T}$ be an infinitely countable set of items we call
    \emph{type variables}.

    For obvious reasons  $\{ \tot, ), ( \} \cap \mathbb{T} = \emptyset$.
\end{defn}

\begin{defn}
    A set $T \subseteq \mathbb{T}$ is called a \emph{set of basic types}
    if it is finite and forms a partial meet-semilattice.
\end{defn}

\begin{defn}
    For a set of basic types $T$, its \emph{type closure}
    $\tclos{T} \subseteq (T \, \dot\cup \,\{\, \tot, ), ( \,\})^*$ is defined
    inductively:

    \begin{itemize}
        \item $\sigma \in T \implies \sigma \in \tclos{T}$
        \item $\sigma, \tau \in \tclos{T} \implies (\sigma \tot \tau) \in \tclos{T}$
    \end{itemize}

    Moreover, we can extend $\less$ over
    $\tclos{T}$:
    \begin{itemize}
        \item $(\sigma \tot \sigma') \less (\tau \tot \tau') \iff (\sigma \less \sigma')
            \& (\tau \less \tau')$
    \end{itemize}
\end{defn}

\begin{prop}
    If $T$ is a set of basic types, then $\tclos{T}$ is a partial meet-semilattice.
\end{prop}

\begin{defn}
    We call a set $X$ \emph{typed in} $T$ $\iff$ there is a function
    \[ typeof : X \rightarrow \tclos{T} \] and $T$ is a partial meet-semilattice.
\end{defn}

\subsubsection{Terms}
To define $\lambda$ terms with the types introduced above, a system similar
to Church's $\lambda_\rightarrow$ \cite[chap.~2.4]{ttfp} shall be used. It
extends the original system with constants and constructors.

\begin{defn}
    Let $\mathbb{V}$ be an infinite countable set. We shall call its elements
    \emph{variables}.
\end{defn}

\begin{defn}
    Let $C$ be a countable set, typed in $T$,
    whose elements we call \emph{constants}.

    For a set of basic types $T$, we can define the set of
    \emph{pre-typed $\lambda$-terms} ($\Lambda_T^C$).

    \begin{itemize}
        \item Constant:    \[ c \in C \implies c \in \Lambda_T^C \]
        \item Variable:    \[ v \in \mathbb{V} \implies v \in \Lambda_T^C \]
        \item Application: \[ A, B \in \Lambda_T^C \implies (AB) \in \Lambda_T^C \]
        \item Abstraction: \[ v \in \mathbb{V}, A \in \Lambda_T^C, \sigma \in \tclos{T}
                \implies (\lambda v : \sigma \abstr A) \in \Lambda_T^C \]
        \item Construction: \[ \sigma \in T, A \in \Lambda_T^C
                \implies \sigma[A] \in \Lambda_T^C \]
    \end{itemize}
\end{defn}

\begin{defn}
    Statement, declaration, context, judgement
    \begin{itemize}
        \item For $M \in \Lambda_T^C, \sigma \in \tclos{T}$, $M : \sigma$ is called
            a \emph{statement}. $M$ is called a \emph{subject} and $\sigma$
            is called a \emph{type}.
        \item A statement with a variable as a subject is called a \emph{declaration}.
        \item A set of declarations with different subjects is called a \emph{context}.
        \item A \emph{judgement} has the form $\Gamma \vdash M: \sigma$, where
            $\Gamma$ is a context and $M: \sigma$ is a statement.
    \end{itemize}

    Moreover, appending to contexts is defined as follows:
    \[ \Gamma \circ x : \sigma = \{ y : \tau \in \Gamma \mid y \neq x \}
       \cup \{ x : \sigma \} \]
\end{defn}

\begin{defn}
    To define what it means for a term $M \in \Lambda_T^C$ to have a type,
    we will use derivation rules.
    \begin{itemize}
        \item Constant
            \cenderiv{
                \hypo{c \in C}
                \infer1{\Gamma \vdash c : typeof(c)}
            }
        \item Variable
            \cenderiv{
                \hypo{x : \sigma \in \Gamma}
                \infer1{\Gamma \vdash x : \sigma}
            }
        \item Application
            \cenderiv{
                \hypo{\Gamma \vdash A : \sigma' \tot \sigma''}
                \hypo{\Gamma \vdash B : \tau}
                \hypo{\tau \less \sigma'}
                \infer3{\Gamma \vdash (AB) : \sigma''}
            }
        \item Abstraction
            \cenderiv{
                \hypo{\Gamma \circ x : \tau \vdash A : \sigma}
                \infer1{\Gamma \vdash (\lambda x : \tau \abstr A) : \tau \tot \sigma}
            }
        \item Construction via upcast
            \cenderiv{
                \hypo{\Gamma \vdash A : \sigma}
                \hypo{\sigma \less \tau}
                \infer2{\Gamma \vdash \tau[A] : \tau}
            }
        \item Construction via downcast (dangerous, doesn't need to be included in most cases)
            \cenderiv{
                \hypo{\Gamma \vdash A : \sigma}
                \hypo{\tau \less \sigma}
                \infer2{\Gamma \vdash \tau[A] : \tau}
            }
    \end{itemize}

    Now, $typeof(M)$ for $M \in \Lambda_T^C$ is defined as such:
    \[
        typeof(M) =
        \begin{cases*}
            \sigma & if $\varnothing \vdash M : \sigma$ \\
            \neg ! & otherwise
        \end{cases*}
    \]
    Of course, the above function is nowhere defined for non-closed terms.
\end{defn}

\subsubsection{Subtype libraries}
\begin{defn}
    A statement in the form $\sigma \lass \tau (\sigma, \tau \in \mathbb{T})$
    is called a \emph{subtype assertion}.
\end{defn}

\begin{defn}
    If we have
    \begin{itemize}
        \item a set of basic types $T$
        \item $\tau \in \tclos{T}$
        \item $\sigma \in \mathbb{T} \setminus T$
    \end{itemize}
    We can define \[ T' = T \tadd (\sigma \lass \tau) \] as:
    \begin{itemize}
        \item If $\tau \in T$, then:
            \begin{itemize}
                \item $T' = T \cup \{ \sigma \}$
                \item $\less$ is extended with $\sigma \less \tau$
            \end{itemize}
        \item If $\tau = \tau' \tot \tau''$, then, for the fresh type variables $\sigma', \sigma''$:
            \begin{itemize}
                \item $T' = T \tadd \sigma' \lass \tau' \tadd \sigma'' \lass \tau''$
            \end{itemize}
    \end{itemize}

    Moreover, if we have an injective function $\varphi : X \rightarrow \tclos{T}$ and
    $T' = T \tadd (\sigma \lass \tau)$, we can define
    \[\varphi' = \varphi \tadd (\sigma \lass \tau) : X \cup \sigma \rightarrow \tclos{T'}\]
    as:
    \begin{itemize}
        \item If $\tau \in T$, then:
            \[ \varphi' = \varphi \cup \{ (\sigma, \sigma) \} \]
        \item If $\tau = \tau' \tot \tau'', T \tadd \sigma \lass \tau
            = T \tadd \sigma' \lass \tau' \tadd \sigma''$:
            \[ \varphi' = \varphi \cup \{ (\sigma, \sigma' \tot \sigma'') \} \]
    \end{itemize}
\end{defn}

\begin{defn}
    A finite sequence of subtype assertions
    $\Theta = \theta_1 ... \theta_n$ is called a \emph{subtype library}
    if $T_{\Theta} = T \oplus \theta_1 \oplus \theta_2 ... \oplus \theta_n$ is defined.
\end{defn}
\end{document}
